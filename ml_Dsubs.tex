\section{\Dsubs{} selections with machine learning}
For the \Dsubs{} analysis the possibility to apply a machine learning approach to the particle candidate selection, 
replacing the linear selections traditionally used, is explored. The objective of this study is to improve the \Dsubs{} 
signal extraction and in particular to look for an extension of the \(\RAA{}\) measurement at low \pt{}, between 2 and 3 \GeV{}.

\subsection{Introduction}
The term \textit{machine learning} comprehends a vast landscape of algorithms which can ``learn'' to perform a task without being explicitly programmed,
these methods are well suited for classification and regression problems. When the problem of interest consists in the discrimination between two or more classes 
usually \textit{supervised} models are employed. \\ 
Supervised algorithms learn how to correctly classify the data starting from a set of examples, of which the belonging class is known. This is referred as the \textit{training set}. 
In a procedure called \textit{training}, which is different for every algorithm, the internal parameters of the model are fixed using the training set. 
The objective of this operation is to obtain the parameters configuration that leads to the best discrimination of the training data. 
Ideally, the model learns some general patterns which are related to the different class typologies and not to the single sample them self. 
Therefore, after the training, the model can be employed to label unknown samples. \\
The output of a supervised model is a \textit{score} evaluated from the instance properties, normally called \textit{features}. 
This score is a numerical value related to the instance probability of belonging to the different classes. 
In order to define which instances are members of a class and which of the other, it is necessary to select a \textit{threshold} value. 
This delimiting value could be defined in relation to the analysis characteristics and requirements; e.g. the relative abundances of the two classes, 
the background rejection or the signal selection efficiency needed. \\
Machine learning models can apply more complex and refined selections with respect to linear cuts. In fact, they can account 
for non-linear relations between the variables used in the discrimination. Moreover, while in the standard approach 
a candidate must satisfy all the conditions imposed to be selected, some characteristics of a real \Dsubs{} candidate can drive an 
high output of the machine learning algorithm even if not all of them fulfil the conditions that would be imposed with the linear selections. 
Thus, leading to the selection of a signal candidate that would be rejected with the standard approach. \\
The aforementioned factors should lead to a better separation of signal and background and a higher selection efficiency.

\subsection{Model used and software}
The machine learning technique used in the analysis is the Boosted Decision Trees (BDT) which showed satisfactory results in analyses similar to this one.
In particular the BDT provided by the XGBoost library \cite{DBLP:journals/corr/ChenG16} is employed. 
This library is an open-source package and its algorithms achieved state-of-the-art results on many machine learning challenges, such as those hosted on Kaggle \cite{Kaggle}.
The model training and performance assessment are performed locally exploiting the software available in the MLHEP package \cite{MLHEP}. 
While the inference step on the experimental data is done on the GRID, the \Dsubs task \texttt{AliAnalysisTaskSEDs.cxx} has been modified to support the loading and application of 
a BDT on particle candidates.

\subsection{Data sample}
The data sample employed is the one used in the standard analysis, i.e. using the linear selections, described in Section~\ref{sec:data_sample}.
Regarding the MC sample, in addition to the one used in the standard analysis (Section~\ref{sec:mc_sample}), a dedicated production enriched in 
\Dsubs candidates is employed. The related production cycles are \texttt{LHC19d4a} and \texttt{LHC19d4b}, which correspond to central (0-10\%) 
and semi-central (30-50\%) events respectively. These MC samples are generated with HIJING + HF Pythia events. More details and the results of
the QA are reported in the JIRA ticket ALIROOT-8249. \\ 

\subsubsection{Pre-selections}
The candidates, considered in the machine learning analysis, have daughters that satisfy the single-track selections reported in Section~\ref{sec:single_track}.
For semi-central events the requirement of at least 70 (out of a maximum of 159) track-associated space points in the TPC is released to 50.
Less restrictive particle identification requirements are imposed with respect to those of the standard analysis, introduced in Section~\ref{sec:pid_sel}. 
In particular, the $2\sigma$ compatibility required for tracks with hits only in the TPC and $\pt{} \leq 8~\GeV{}$ is relaxed to $3\sigma$. 
This allow us to train the model also employing variables related to the PID and let it perform a finer discrimination. 
Finally, loose topological and kinematic selections are applied in order to reduce the amount of candidates in the data samples. They are reported in 
Table~\ref{tab:dsubs_pre} and Table~\ref{tab:dsubs2_pre} for the 0-10\% and 30-50\% centrality classes respectively.
These pre-selections essentially reject the background and preserve a relevant fraction of the signal candidates that pass the single-track selections.
\begin{table}[h!]
  \begin{center}
    \caption{Topological and kinematic pre-selections applied for the \Dsubs analysis in 0--10\% centrality class.}
    \label{tab:dsubs_pre}
    \begin{tabular}{|c|c|c|}
      \hline
      \textbf{$\mathbf{\pt{}}$ (GeV/\textit{c})} & \textbf{[2,5]} & \textbf{[5,50]} \\
      \hline
      \pt($\pi$, $K$) (GeV/$c$) & $>$ 0.6 & $>$ 0.6  \\
      \hline
      Dec. length ($\mum$)  & $>$ 500 & $>$ 200 \\
      \hline
      Norm. Dec. length XY  & $>$ 5 & $>$ 2  \\
      \hline
      Cosine pointing  & $>$ 0.97 & $>$ 0.90 \\
      \hline
      Cosine pointing XY & $>$ 0.96 & $>$ 0.90 \\
      \hline
      $\sigma_{vertex}$  (cm) &  $<$ 0.04 & $<$ 0.06 \\
      \hline
      M$^{\phi}_{inv}$ - M$^{\phi PDG}_{inv}$ (MeV/$c^{2}$) & $<$ 10 & $<$ 15  \\
      \hline
    \end{tabular}
  \end{center}
\end{table}

\begin{table}[h!]
  \begin{center}
    \caption{Topological and kinematic pre-selections applied for the \Dsubs analysis in 30--50\% centrality class.}
    \label{tab:dsubs2_pre}
    \begin{tabular}{|c|c|c|}
      \hline
      \textbf{$\mathbf{\pt{}}$ (GeV/\textit{c})} & \textbf{[2,5]} & \textbf{[5,36]} \\
      \hline
      \pt($\pi$, $K$) (GeV/$c$) & $>$ 0.4 & $>$ 0.4  \\
      \hline
      Dec. length ($\mum$)  & $>$ 200 & $>$ 200 \\
      \hline
      Norm. Dec. length XY  & $>$ 2 & $>$ 2  \\
      \hline
      Cosine pointing  & $>$ 0.96 & $>$ 0.90 \\
      \hline
      Cosine pointing XY & $>$ 0.96 & $>$ 0.90 \\
      \hline
      $\sigma_{vertex}$  (cm) &  $<$ 0.045 & $<$ 0.06 \\
      \hline
      M$^{\phi}_{inv}$ - M$^{\phi PDG}_{inv}$ (MeV/$c^{2}$) & $<$ 15 & $<$ 15  \\
      \hline
    \end{tabular}
  \end{center}
\end{table} 

\subsubsection{Train and test set}
In order to obtain a machine learning algorithm able to make predictions, it is necessary to build the data sets on which the model training is 
performed and its performance is evaluated. The training and test set are constructed from MC and real data samples. \\
The training and test set are composed by the prompt \Dsubs{} and charge conjugate candidates available in the MC sample for the signal. 
The \texttt{LHC19d4} productions, enhanced in \Dsubs{} candidates, are used for the training set while the \texttt{LHC19c3} productions are used
for the test set.
The background, instead, is extracted from the data sample of real candidates. This is done selecting \Dsubs{} and charge conjugate candidates 
in the side bands of the the invariant mass distribution, away from the \Dplus{} and \Dsubs{} peaks. The idea behind this decision is to use real data, where feasible,
to avoid a loss in the model predictive power caused by possible shortcomings of simulations in describing the real events. \\
The number of signal and background candidates used to build the training set is reported, for the different \pt{} regions in which the training is performed, 
in Table~\ref{tab:data_avail_010} and Table~\ref{tab:data_avail_3050} for the 0-10\% and 30-50\% centrality classes respectively. The signal candidates 
are all those available in the \texttt{LHC19d4} productions, while the number of background candidates is chosen to be roughly the double of the signal. 
For the low-\pt{} central collisions more than the double of candidates for the background are used due to the low amount of signal candidates. At high \pt{} less 
background candidates are used to maintain small the fraction of real data employed in the training with respect to the total amount available.
The test set is composed by all the \Dsubs{} candidates available in the \texttt{LHC19c3} productions and the same amount of background candidates
as the training set. \\ 
\begin{table}[h!]
  \begin{center}
  \caption{Number of signal and background candidates used in the training set, 0--10\% centrality class.}
  \label{tab:data_avail_010}
    \begin{tabular}{|c|c|c|}
      \hline
      \textbf{$\mathbf{\pt{}}$ (GeV/\textit{c})} &  \textbf{\# signal} & \textbf{\# background} \\
      \hline
      \(2 - 3\) & \(\sim 9.6 \cdot 10^3\) & \(50 \cdot 10^3\) \\
      \hline
      \(3 - 4\) & \(\sim 21 \cdot 10^3\) & \(80 \cdot 10^3\) \\
      \hline
      \(4 - 6\) & \(\sim 67 \cdot 10^3\) & \(160 \cdot 10^3\) \\
      \hline
      \(6 - 12\) & \(\sim 204 \cdot 10^3\) & \(200 \cdot 10^3\) \\
      \hline
      \(12 - 50\) & \(\sim 242 \cdot 10^3\) & \(120 \cdot 10^3\) \\
      \hline
    \end{tabular}
  \end{center}
\end{table}
\begin{table}[h!]
  \begin{center}
  \caption{Number of signal candidates available to build the training test set, 30--50\% centrality class.}
  \label{tab:data_avail_3050}
    \begin{tabular}{|c|c|c|}
    \hline
    \textbf{$\mathbf{\pt{}}$ (GeV/\textit{c})} &  \textbf{\# signal} & \textbf{\# background} \\
    \hline
    \(2 - 3\) & \(\sim 50 \cdot 10^3\) & \(100 \cdot 10^3\) \\
    \hline
    \(3 - 4\) & \(\sim 68 \cdot 10^3\) & \(140 \cdot 10^3\) \\
    \hline
    \(4 - 6\) & \(\sim 122 \cdot 10^3\) & \(240 \cdot 10^3\) \\
    \hline
    \(6 - 12\) & \(\sim 216 \cdot 10^3\) & \(220 \cdot 10^3\) \\
    \hline
    \(12 - 36\) & \(\sim 226 \cdot 10^3\) & \(20 \cdot 10^3\) \\
    \hline
    \end{tabular}
  \end{center}
\end{table}

\subsection{Model training}
\subsubsection{Training varaibles}
\label{sec:train_variables}
Topological and kinematic variables are used in the model training together with the particle identification 
information of each candidate daughter, for which it has been chosen to combine the TPC and TOF information as follow
\begin{equation}
  n\sigma_{comb}^{\pi,K} = 
  \begin{cases}
    |n\sigma_{TPC}^{\pi,K}| & \text{tracks only in TPC} \\
    \\[4pt]
    |n\sigma_{TOF}^{\pi,K}| & \text{tracks only in TOF} \\
    \\[4pt]
    \frac{1}{\sqrt{2}}\sqrt{\big(n\sigma_{TPC}^{\pi,K}\big)^2 + \big(n\sigma_{TOF}^{\pi,K}\big)^2} & \text{tracks in TPC and TOF}
  \end{cases} 
\end{equation}
where \(n\sigma^{\pi,K}\) is computed from the specific energy loss $dE/dx$ (for the TPC) and the time of flight (for the TOF) 
measured for a track. The difference between the measured and expected signal for the corresponding particle hypothesis 
is considered and then it is expressed in units of the resolution as follow
\begin{equation}
  n\sigma^{\pi,K} = \frac{|S_{meas} - S_{exp}^{\pi,K}|}{\sigma^{\pi,K}}
\end{equation}
where $\sigma^{\pi,K}$ is the resolution on the energy loss or the time of flight signals for each species. 
This has been done to maintain low the number of variables used in the training and to account for tracks that do not have
hits in the TOF detector. In the end six PID variables are obtained, two for each track, which contain the information on the 
compatibility of the track with the pion and kaon hypothesis.\\

The training variables used are reported in Table~\ref{tab:train_var} and their distributions for signal and background, 
in the \pt{} region between 2 and 3 \(\GeV{}\), are showed in Figure~\ref{fig:var_distr_010} and \ref{fig:var_distr_3050}
for central and semi-central collision respectively. Also the distributions of the particle candidate mass (\texttt{inv\_mass}) 
and transverse momentum (\texttt{pt\_cand}), which are not used in the training, are included.
\begin{table}[htpb]
  \begin{center}
  \caption{Training variables used, the names that are used in the plots are reported in parenthesis.}
  \label{tab:train_var}
    \begin{tabular}{|l|l|}
      \hline
      Decay length & (\texttt{d\_len}) \\
			\hline
			Normalized decay length XY & (\texttt{norm\_dl\_xy}) \\
			\hline
			Cosine pointing angle XY & (\texttt{cos\_p\_xy}) \\
			\hline
			$\sigma_{vertex}$ & (\texttt{sig\_vert}) \\
			\hline
			M$^{\phi}_{inv}$ - M$^{\phi PDG}_{inv}$ & (\texttt{delta\_mass\_KK}) \\
			\hline
			$|\cos^3 \theta^\prime({\rm K})|$ & (\texttt{cos\_PiKPhi\_3}) \\
			\hline
			$|d_0-d_0^{exp}|^{prong}(n\sigma)$ & (\texttt{max\_norm\_d0d0exp}) \\
      \hline
      $n\sigma_{comb}^{\pi}(0)$ & (\texttt{nsigComb\_Pi\_0})\\
      \hline
      $n\sigma_{comb}^{\pi}(1)$ & (\texttt{nsigComb\_Pi\_1})\\
      \hline
      $n\sigma_{comb}^{\pi}(2)$ & (\texttt{nsigComb\_Pi\_2})\\
      \hline
      $n\sigma_{comb}^{K}(0)$ & (\texttt{nsigComb\_K\_0})\\
      \hline
      $n\sigma_{comb}^{K}(1)$ & (\texttt{nsigComb\_K\_1})\\
      \hline
      $n\sigma_{comb}^{K}(2)$ & (\texttt{nsigComb\_K\_2})\\
      \hline
    \end{tabular}
  \end{center}
\end{table}
\begin{figure}[htpb]
  \begin{center}
   \includegraphics[width=1\textwidth]{figures/Dsubs/ml/variablesDistribution_010_pt2_3}
   \caption{Signal and background variable distributions for candidates in the 0-10\% centrality class.}
  \label{fig:var_distr_010}
  \end{center}
\end{figure}
\begin{figure}[htpb]
  \begin{center}
   \includegraphics[width=1\textwidth]{figures/Dsubs/ml/variablesDistribution_3050_pt2_3}
   \caption{Signal and background variable distributions for candidates in the 30-50\% centrality class.}
  \label{fig:var_distr_3050}
  \end{center}
\end{figure}
The linear correlations between the variables used in the training as well as the invariant mass of the \Dsubs{} candidate and 
its transverse momentum, which are not employed in the training, for signal and background are presented 
in Figure \ref{fig:corr_010} and Figure \ref{fig:corr_3050} for the two centrality classes considered in the analysis.
\begin{figure}[htbp]
  \begin{center}
   \includegraphics[width=0.45\textwidth]{figures/Dsubs/ml/correlationmatrixsignal_010_pt2_3.png}
   \includegraphics[width=0.45\textwidth]{figures/Dsubs/ml/correlationmatrixbackground_010_pt2_3.png} 
  \caption{Linear correlations between the training variables, the invariant mass of the \Dsubs{} candidates and its transverse 
  momentum, for signal (left) and background (right) candidates in the 0-10\% centrality class. The colours, 
  from red (fully correlated) to blue (fully anti-correlated), indicate the correlation level.}
  \label{fig:corr_010}
  \end{center}
\end{figure}
\begin{figure}[htbp]
  \begin{center}
   \includegraphics[width=0.45\textwidth]{figures/Dsubs/ml/correlationmatrixsignal_3050_pt2_3.png}
   \includegraphics[width=0.45\textwidth]{figures/Dsubs/ml/correlationmatrixbackground_3050_pt2_3.png} 
  \caption{Linear correlations between the training variables, the invariant mass of the \Dsubs{} candidates and its transverse 
  momentum, for signal (left) and background (right) candidates in the 30-50\% centrality class. The colours, 
  from red (fully correlated) to blue (fully anti-correlated), indicate the correlation level.}
  \label{fig:corr_3050}
  \end{center}
\end{figure}
Variables that carry the same physical informations, such as those related to the decay length, are strongly correlated 
as expected. Moreover, there are some differences in the variable correlations between signal and background candidates, 
which could be exploited by the model to discriminate signal from background. It is also useful to control the presence of 
correlations between the training variables, the invariant mass and the \pt of the candidates. 
In particular, it is preferable that the selection based on the model predictions does not modify the invariant mass distributions too much. 
In fact, a complex shape of the background invariant mass distribution requires a function with 
more free parameters to be described, therefore the yield extraction is more affected by eventual background fluctuations. \\
No variables are strongly correlated with the invariant mass, thus no deformation of the background invariant mass distribution or  
formation of structures are expected.

\subsubsection{Training and performance evaluation}
The number of signal and background candidates used in the training set are different, as shown in Table~\ref{tab:data_avail_010} 
and~\ref{tab:data_avail_3050}. This choice has been done because a training set with more examples improve the model 
performance and while the number of signal candidates coming from MC simulations is limited, the background candidates are usually  
available in much higher quantity. The class imbalance introduced, in the case in which the machine learning model used is a 
BDT as in this analysis, does not cause biases in the candidate selection. \\
The hyper-parameter configuration used for the training, in the different \pt{} bins considered, is reported in Table~\ref{tab:hyp_par} 
and is common for both centrality classes. The complete list of model hyper-parameters, together with their default values and a description 
of their meaning can be found in the XGBoost documentation~\cite{XGBdoc, XGBdefault}. 
\begin{table}[h!]
  \begin{center}
  \caption{Hyper-parameter configuration for the \pt{} bin considered in the training.}
  \label{tab:hyp_par}
    \begin{tabular}{|c|c|c|}
      \hline
      \textbf{Hyper-parameter} & \textbf{\(\mathbf{\pt{} < 6}\) (GeV/\textit{c})} & \textbf{\(\mathbf{\pt{} > 6}\) (GeV/\textit{c})} \\
      \hline
      \texttt{learning\_rate} & 0.1 & 0.1 \\
      \hline
      \texttt{max\_depth} & 4 & 5 \\
      \hline
      \texttt{min\_child\_weight} & 2 & 2 \\
      \hline
      \texttt{colsample\_bytree} & 0.9 & 0.9 \\
      \hline
      \texttt{tree\_method} & 'approx' & 'approx' \\
      \hline
    \end{tabular}
  \end{center}
\end{table}
The number of trees that compose the ensemble is fixed using the early stopping method and a 5-fold cross-validation \cite{stat_learn} 
to estimate the model performance as the number of trees in the ensemble increases. It has been observed that this approach gives a resonable 
tuning of the model complexity without the need of an extensive and time consuming search over all the hyper-parameters. 
The number of trees, \texttt{n\_estimators}, selected for the different \pt{} intervals in which the training is performed are reported in 
Table~\ref{tab:numtree} for the 0-10\% and 30-50\% centrality classes. \\
\begin{table}[h!]
  \begin{center}
  \caption{Number of trees used for the \pt{} intervals in which the training is performed, 0--10\% and 30--50\% centrality class.}
  \label{tab:numtree}
    \begin{tabular}{|c|c|c|}
      \hline
      \textbf{$\mathbf{\pt{}}$ (GeV/\textit{c})} &  \textbf{\texttt{n\_estimators} (0-10)\%} & \textbf{\texttt{n\_estimators} (30-50)\%} \\
      \hline
      \(2 - 3\) & 300 & 378 \\
      \hline
      \(3 - 4\) & 427 & 460 \\
      \hline
      \(4 - 6\) & 625 & 969 \\
      \hline
      \(6 - 12\) & 1393 & 1349 \\
      \hline
      \(12 - 50\) & 1167 & 527 \\
      \hline
    \end{tabular}
  \end{center}
\end{table}

After the training, to assess the performance of the model the area under the Receiver Operating Characteristic curve (ROC AUC) 
is considered. The Receiver Operating Characteristic (ROC) curve is obtained plotting the signal selection efficiency, the 
True Positive Rate in more general terms, as function of the background selection efficiency, False Positive Rate, for various 
threshold settings on the model output. The possible values of the ROC AUC are comprehended between one half and one where 0.5 corresponds 
to a random classification and 1 to a perfect discrimination of the two class samples. The ROC AUC gives a global estimation of the model 
performance, i.e. not related to the threshold value that will be chosen. Moreover, it is independent with respect to the
relative abundances of the class instances in the dataset used for its evaluation. \\
The ROC curves and the values of ROC AUC, obtained on the training and test set, for the models trained in the 2-3 \pt{} region are reported in Figure~\ref{fig:roc_pt2_3} 
for central and semi-central events.
\begin{figure}[htbp]
  \begin{center}
   \includegraphics[width=0.48\textwidth]{figures/Dsubs/ml/ROC_Curve_0_10_pt2_3}
   \includegraphics[width=0.48\textwidth]{figures/Dsubs/ml/ROC_Curve_30_50_pt2_3} 
  \caption{ROC curves for the models trained in the 2-3 \pt{} region in the 0-10\% (left) and 30-50\% (right) centrality classes. 
           The results obtained on the training set are reported in light blue while those obtained on the test set are in dark blue.}
  \label{fig:roc_pt2_3}
  \end{center}
\end{figure}
A small discrepancy between the ROC curves on the training and test set indicates that the model is not over-fitting. A similar behaviour 
is observed in the other \pt{} regions in which the training is performed and the ROC AUC on the test set range between \(\sim 0.97\) and \(\sim 0.985\).


\subsection{Model application on data}
From the machine learning model application on the real-data candidate a `BDT output' comprehended between 0 and 1 is obtained,
this value can be interpreted as the probability of a particle candidate to be a signal one. It is necessary to choose a 
threshold value on the BDT output, in order to reject a large number of background candidates and keep most of the \Dsubs{} meson signal. 
Thus a scan of the possible BDT-output threshold values is performed. 
For each threshold value the BDT-selection efficiency of prompt \Dsubs{} is computed exploiting the test set. 
Moreover, an expected signal significance for the given threshold is estimated. To this end the expected signal is computed 
multiplying the pre-selection and BDT-selection efficiencies to the predicted yield obtained with FONLL and TAMU. 
While the expected background under the signal peak is estimated from a fit to the sidebands of the invariant mass distribution of 
a small portion of real data after the application of the BDT selection. It has been found that the values of expected significance
predicted are very sensitive to background fluctuations and to the \pt{} shape of the signal. Therefore, they aren't used directly for the 
cut selection but are useful to have an idea of the region of interest for the threshold. \\
The threshold value is tuned to preserve the highest possible selection efficiency and a good statistical significance and stability of the \Dsubs{}-meson signal extraction. 
The prompt BDT-selection efficiency as a function of the threshold value is reported, together with the expected significance, 
for the transverse momentum interval within 2 and 3 \(\GeV{}\) in Figure~\ref{fig:effsig_pt2_3_010} and~\ref{fig:effsig_pt2_3_3050}, for the 0-10\% and 30-50\% centrality classes respectively.
\begin{figure}[htbp]
  \begin{center}
   \includegraphics[width=0.48\textwidth]{figures/Dsubs/ml/EfficiencyCen010_pt2_3}
   \includegraphics[width=0.48\textwidth]{figures/Dsubs/ml/ExpSignif_0_10_pt2_3} 
  \caption{Prompt BDT-selection efficiency (left) and expected significance (right) in the 2-3 \pt{} region for the 0-10\% centrality class.}
  \label{fig:effsig_pt2_3_010}
  \end{center}
\end{figure}
\begin{figure}[htbp]
  \begin{center}
    \includegraphics[width=0.48\textwidth]{figures/Dsubs/ml/EfficiencyCen3050_pt2_3}
    \includegraphics[width=0.48\textwidth]{figures/Dsubs/ml/ExpSignif_30_50_pt2_3} 
  \caption{Prompt BDT-selection efficiency (left) and expected significance (right) in the 2-3 \pt{} region for the 30-50\% centrality class.}
  \label{fig:effsig_pt2_3_3050}
  \end{center}
\end{figure}

\subsection{Signal extraction}
The models trained in the different \pt{} intervals are applied to the respective transverse momentum region. The threshold values used, 
as a function of \pt{}, are reported in Table~\ref{tab:cut_values} for the 0-10\% and 30-50\% centrality classes. \\
\begin{table}[h!]
  \begin{center}
  \caption{Threshold values on the BDT output for the different \pt{} bins of the analysis, for the 0-10\% and 
          30-50\% centrality classes}
  \label{tab:cut_values}
  \begin{tabular}{|c|c|c|}
    \hline
    \textbf{$\mathbf{\pt{}}$ (GeV/\textit{c})} & \textbf{Threshold value (0-10\%)} & \textbf{Threshold value (30-50\%)} \\
    \hline
    \(2 - 3\) & 0.9925 & 0.9983 \\
    \hline
    \(3 - 4\) & 0.9950 & 0.9980 \\
    \hline
    \(4 - 5\) & 0.9983 & 0.9970 \\
    \hline
    \(5 - 6\) & 0.9975 & 0.9965 \\
    \hline
    \(6 - 8\) & 0.9985 & 0.997 \\
    \hline
    \(8 - 12\) & 0.9965 & 0.985 \\
    \hline
    \(12 - 16\) & 0.96 & 0.975 \\
    \hline
    \(16 - 24\) & 0.94 & 0.975 \\
    \hline
    \(24 - 36\) & 0.98 & 0.985 \\
    \hline
    \(36 - 50\) & 0.955 & --- \\
    \hline
  \end{tabular}
  \end{center}
\end{table}

The signal extraction procedure is the same as that reported in Section~\ref{sec:sign_extr}. 
The invariant mass distributions obtained in the 0-10\% and 30-50\% centrality classes are reported in Figure~\ref{fig:sig_ml_010} and~\ref{fig:sig_ml_3050} respectively. \\
\begin{figure}[htpb]
  \begin{center}
   \includegraphics[width=1\textwidth]{figures/Dsubs/ml/RawYieldsDs_010_pt2_50.pdf}
   \caption{\Dsubs{} signal in 0--10\% in the range \(2 < \pt{} < 50 \GeV{}\).}
  \label{fig:sig_ml_010}
  \end{center}
\end{figure}
\begin{figure}[htpb]
  \begin{center}
   \includegraphics[width=1\textwidth]{figures/Dsubs/ml/RawYieldsDs_3050_pt2_36.pdf}
   \caption{\Dsubs{} signal in 30--50\% in the range \(2 < \pt{} < 36 \GeV{}\).}
  \label{fig:sig_ml_3050}
  \end{center}
\end{figure}
The goodness of the mass fit expectations was checked against the Monte Carlo and the standard analysis in terms of mass peak \(\sigma\) and position. 
The comparisons of the Gaussian width and mean of the \Dsubs{}-meson mass peaks in data, with the two different approaches to the candidate selection, 
and Monte Carlo are reported in Figure~\ref{fig:comp_par_010} and~\ref{fig:comp_par_3050} for the 0-10\% and 30–50\% centrality classes respectively. \\
\begin{figure}[htpb]
  \begin{center}
    \includegraphics[width=0.48\textwidth]{figures/Dsubs/ml/Mean_compMCStd_010}
    \includegraphics[width=0.48\textwidth]{figures/Dsubs/ml/Sigma_compMCStd_010}
   \caption{Comparison of Gaussian mean (left) and width (right) extracted from the invariant-mass fits of \Dsubs{} candidates in 0--10\% for the data and the MC simulation.}
  \label{fig:comp_par_010}
  \end{center}
\end{figure}
\begin{figure}[htpb]
  \begin{center}
    \includegraphics[width=0.48\textwidth]{figures/Dsubs/ml/Mean_compMCStd_3050}
    \includegraphics[width=0.48\textwidth]{figures/Dsubs/ml/Sigma_compMCStd_3050}
   \caption{Comparison of Gaussian mean (left) and width (right) extracted from the invariant-mass fits of \Dsubs{} candidates in 30--50\% for the data and the MC simulation.}
  \label{fig:comp_par_3050}
  \end{center}
\end{figure}
The significances per event obtained with the BDT selections are compared to those of the standard analysis in Figure~\ref{fig:signif_comp} 
for the two centrality classes. To perform the comparison, the width of the \Dsubs{} peak has been fixed in the invariant-mass fits to the 
values obtained from the standard analysis. Employing the BDT selections \(\sim50\%\) higher significances per event are obtained with respect to the standard 
analysis in all the comparable \pt{} region and for both centrality classes.
\begin{figure}[htbp]
  \begin{center}
   \includegraphics[width=0.8\textwidth]{figures/Dsubs/ml/SignificancePerEventComparison_ML_Std_010}
   \includegraphics[width=0.8\textwidth]{figures/Dsubs/ml/SignificancePerEventComparison_ML_Std_3050} 
  \caption{Comparison of efficiency$\times$acceptance for prompt (red) and feed-down (blue) \Dsubs{} with BDT selections, 
           prompt (orange) and feed-down (green) \Dsubs{} with linear selections in 0--10$\%$ (left) and 30--50$\%$ (right).}
  \label{fig:signif_comp}
  \end{center}
\end{figure}

\subsection{Selection efficiencies}
The selection efficiency is estimated using the \texttt{LHC19c3a} and \texttt{LHC19c3b} MC productions following the same 
procedure as described in Section~\ref{sec:eff}. 
The prompt and feed-down efficiency$\times$acceptance as a function of \pt{} obtained in the 0-10\% and 30-50\% centrality classes 
are reported in Figure~\ref{fig:eff_tot}.
\begin{figure}[htbp]
  \begin{center}
   \includegraphics[width=0.48\textwidth]{figures/Dsubs/ml/EffAcc_Ds010_pt2_50}
   \includegraphics[width=0.48\textwidth]{figures/Dsubs/ml/EffAcc_Ds3050_pt2_36} 
  \caption{Transverse momentum dependence of efficiency$\times$acceptance for prompt (red) 
           and feed-down (blue) \Dsubs in 0--10$\%$ (left) and 30--50$\%$ (right).}
  \label{fig:eff_tot}
  \end{center}
\end{figure}
The selection efficiencies obtained with the BDT selections are compared to those of the standard analysis in Figure~\ref{fig:eff_comp} 
for the two centrality classes. Employing the BDT selection higher efficiencies are obtained with respect to the standard analysis in all 
the comparable \pt{} region and for both centrality classes.
\begin{figure}[htbp]
  \begin{center}
   \includegraphics[height=6cm, width=7.5cm]{figures/Dsubs/ml/Comp_EffAcc_wStd_010}
   \includegraphics[height=6.05cm, width=7.5cm]{figures/Dsubs/ml/Comp_EffAcc_wStd_3050} 
  \caption{Comparison of efficiency$\times$acceptance for prompt (red) and feed-down (blue) \Dsubs{} with BDT selections, 
           prompt (orange) and feed-down (green) \Dsubs{} with linear selections in 0--10$\%$ (left) and 30--50$\%$ (right).}
  \label{fig:eff_comp}
  \end{center}
\end{figure}

\clearpage

\subsection{Systematic uncertainties}
Given that only the candidate selection step is different from the standard analysis only few of the 
systematic uncertainty contributions need to be re-estimated. These are related to: the raw yield extraction, 
the selection efficiency and the generated \pt{} shape. \\ 
For what concerns the PID selection efficiency, the systematic 
contribution considered is the one related to the conservative PID strategy, used in the pre-selections for the 
dataset of this analysis with the BDT, differently from the standard analysis that use the strong PID at 
\(\pt{} < 8~\GeV{}\). Possible systematics due to the use of the PID variables, described in Section~\ref{sec:train_variables},
in the model training are already accounted in the selection efficiency contribution.
 
\subsubsection{Raw yield extraction}
The procedure for the estimation of the systematic uncertainty due to the raw yield extraction follow what is performed 
in the standard analysis, which is described in Section~\ref{sec:raw_yield_syst}.  The results for \Dsubs{} are shown in 
Figures~\ref{fig:DsYieldSyst010_ml_1},~\ref{fig:DsYieldSyst010_ml_2},~\ref{fig:DsYieldSyst010_ml_3} for the 0-10\% centrality 
class and in Figures~\ref{fig:DsYieldSyst3050_ml_1},~\ref{fig:DsYieldSyst3050_ml_2} for the 30-50\% centrality class.
\begin{figure}[htbp]
  \begin{center}
  \includegraphics[width=0.65\textwidth]{figures/Dsubs/ml/raw_yield_syst_010/RawYieldsSystDs_010_071019_lowpt_pt0}
  \includegraphics[width=0.65\textwidth]{figures/Dsubs/ml/raw_yield_syst_010/RawYieldsSystDs_010_071019_lowpt_pt1_sigmaPlus15}
  \includegraphics[width=0.65\textwidth]{figures/Dsubs/ml/raw_yield_syst_010/RawYieldsSystDs_010_071019_lowpt_pt2}
  \includegraphics[width=0.65\textwidth]{figures/Dsubs/ml/raw_yield_syst_010/RawYieldsSystDs_010_071019_lowpt_pt3}
  \caption{Output of the multi-trial study for \Dsubs mesons in 0--10$\%$. For each \pt bin, the top left panel shows the raw yield 
           as a function of the trial, the top right panel the obtained distribution of raw yields, the bottom left panel the Gaussian 
           sigma as a function of the trial and the bottom right panel the $\chi^2/ndf$ as a function of the trial. In the 3-4 
           \(~\GeV{}/c\) \pt{} bin the trials with fixed sigma are performed with the MC value +15\%.}
  \label{fig:DsYieldSyst010_ml_1}
  \end{center}
\end{figure}
\begin{figure}[htbp]
  \begin{center}
  \includegraphics[width=0.65\textwidth]{figures/Dsubs/ml/raw_yield_syst_010/RawYieldsSystDs_010_071019_lowpt_pt4_sigmaPlus15}
  \includegraphics[width=0.65\textwidth]{figures/Dsubs/ml/raw_yield_syst_010/RawYieldsSystDs_010_071019_highpt_pt5}
  \includegraphics[width=0.65\textwidth]{figures/Dsubs/ml/raw_yield_syst_010/RawYieldsSystDs_010_071019_highpt_pt6}
  \includegraphics[width=0.65\textwidth]{figures/Dsubs/ml/raw_yield_syst_010/RawYieldsSystDs_010_071019_highpt_pt7}
  \caption{Output of the multi-trial study for \Dsubs mesons in 0--10$\%$. For each \pt bin, the top left panel shows the raw yield 
           as a function of the trial, the top right panel the obtained distribution of raw yields, the bottom left panel the Gaussian 
           sigma as a function of the trial and the bottom right panel the $\chi^2/ndf$ as a function of the trial. In the 6-8 
           \(~\GeV{}/c\) \pt{} bin the trials with fixed sigma are performed with the MC value +15\%.}
  \label{fig:DsYieldSyst010_ml_2}
  \end{center}
\end{figure}
\begin{figure}[htbp]
  \begin{center}
  \includegraphics[width=0.55\textwidth]{figures/Dsubs/ml/raw_yield_syst_010/RawYieldsSystDs_010_071019_highpt_pt8}
  \includegraphics[width=0.55\textwidth]{figures/Dsubs/ml/raw_yield_syst_010/RawYieldsSystDs_010_071019_pt36_50_pt9}
  \caption{Output of the multi-trial study for \Dsubs mesons in 0--10$\%$. For each \pt bin, the top left panel shows the raw yield 
           as a function of the trial, the top right panel the obtained distribution of raw yields, the bottom left panel the Gaussian 
           sigma as a function of the trial and the bottom right panel the $\chi^2/ndf$ as a function of the trial.}
  \label{fig:DsYieldSyst010_ml_3}
  \end{center}
\end{figure}
\begin{figure}[htbp]
  \begin{center}
  \includegraphics[width=0.65\textwidth]{figures/Dsubs/ml/raw_yield_syst_3050/RawYieldsSystDs_3050_011019_lowpt_pt0}
  \includegraphics[width=0.65\textwidth]{figures/Dsubs/ml/raw_yield_syst_3050/RawYieldsSystDs_3050_011019_lowpt_pt1}
  \includegraphics[width=0.65\textwidth]{figures/Dsubs/ml/raw_yield_syst_3050/RawYieldsSystDs_3050_011019_lowpt_pt2}
  \includegraphics[width=0.65\textwidth]{figures/Dsubs/ml/raw_yield_syst_3050/RawYieldsSystDs_3050_011019_lowpt_pt3}
  \caption{Output of the multi-trial study for \Dsubs mesons in 30--50\%. For each \pt bin, the top left panel shows the raw yield 
           as a function of the trial, the top right panel the obtained distribution of raw yields, the bottom left panel the Gaussian 
           sigma as a function of the trial and the bottom right panel the $\chi^2/ndf$ as a function of the trial.}
  \label{fig:DsYieldSyst3050_ml_1}
  \end{center}
\end{figure}
\begin{figure}[htbp]
  \begin{center}
  \includegraphics[width=0.65\textwidth]{figures/Dsubs/ml/raw_yield_syst_3050/RawYieldsSystDs_3050_011019_lowpt_pt4}
  \includegraphics[width=0.65\textwidth]{figures/Dsubs/ml/raw_yield_syst_3050/RawYieldsSystDs_3050_011019_lowpt_pt5}
  \includegraphics[width=0.65\textwidth]{figures/Dsubs/ml/raw_yield_syst_3050/RawYieldsSystDs_3050_011019_highpt_pt6}
  \includegraphics[width=0.65\textwidth]{figures/Dsubs/ml/raw_yield_syst_3050/RawYieldsSystDs_3050_011019_highpt_pt7}
  \caption{Output of the multi-trial study for \Dsubs mesons in 30--50\%. For each \pt bin, the top left panel shows the raw yield 
           as a function of the trial, the top right panel the obtained distribution of raw yields, the bottom left panel the Gaussian 
           sigma as a function of the trial and the bottom right panel the $\chi^2/ndf$ as a function of the trial.}
  \label{fig:DsYieldSyst3050_ml_2}
  \end{center}
\end{figure}
The numerical values of the systematic on yield extraction, determined bin by bin for the two centrality classes considered, 
are reported in Tables~\ref{tab:DsYieldSyst010_ml} and ~\ref{tab:DsYieldSyst3050_ml}.
\begin{table}[htbp]
  \begin{center}
   \begin{tabular}{|c|c|}
    \hline
    \textbf{$\mathbf{\pt{}}$ (GeV/\textit{c})} & \textbf{Raw-yield syst. unc.} \\
    \hline
    2-3 & 10\%\\
    \hline
    3-4 & 8\%\\
    \hline
    4-5 & 4\%\\
    \hline
    5-6 & 3\%\\
    \hline
    6-8 & 3\%\\
    \hline
    8-12 & 3\%\\
    \hline
    12-16 & 3\%\\
    \hline
    16-24 & 3\%\\
    \hline
    24-36 & 5\%\\
    \hline
    36-50 & 10\%\\ 
    \hline
   \end{tabular}
  \end{center}
  \caption{Systematic uncertainty from yield extraction for the \Dsubs{} mesons in the 0--10\% centrality class.}
  \label{tab:DsYieldSyst010_ml}
\end{table}
\begin{table}[htbp]
  \begin{center}
   \begin{tabular}{|c|c|}
    \hline
    \textbf{$\mathbf{\pt{}}$ (GeV/\textit{c})} & \textbf{Raw-yield syst. unc.} \\
    \hline
    2-3 & 9\%\\
    \hline
    3-4 & 7\%\\
    \hline
    4-6 & 4\%\\
    \hline
    6-8 & 3\%\\
    \hline
    8-12 & 3\%\\
    \hline
    12-16 & 3\%\\
    \hline
    16-24 & 3\%\\
    \hline
    24-36 & 6\%\\
    \hline
   \end{tabular}
  \end{center}
  \caption{Systematic uncertainty from yield extraction for the \Dsubs{} mesons in the 30--50\% centrality class.}
  \label{tab:DsYieldSyst3050_ml}
\end{table} 

\clearpage
 
\subsubsection{Selection efficiency}
A systematic uncertainty can arise from possible differences in the particle-variable shapes between data and Monte Carlo 
that can affect the model training, thus the final selections, and due to residual misalignment. 
These sources were checked by repeating the analysis varying the threshold on the BDT output from the central value with an approach 
similar to the one described in Section~\ref{sec:eff_syst}. \\
A systematic scan from loose to tight selections was done on the threshold value, reconstructing the yields, comparing them to the yields obtained 
with the reference threshold value and looking for possible trends/biases of the reconstructed yield as a function of the cut strength. 
A set of 30 looser threshold and 20 tighter ones was considered, looking for a \(\pm~50\%\) efficiency 
variation with respect to the central case. \\
The result of this study for each \pt{} bin is shown in Figures~\ref{fig:DsCutSyst010_ml_1},~\ref{fig:DsCutSyst010_ml_2},~\ref{fig:DsCutSyst010_ml_3} 
for central events and in Figures~\ref{fig:DsCutSyst3050_ml_1},~\ref{fig:DsCutSyst3050_ml_2} for semi-central ones. The variations of the raw yield, prompt and
feed-down \Dsubs{} efficiency, significance, signal-to-background ratio, and corrected yield are plotted as a
function of the cut set tested. Finally the distribution of the variation of the corrected yield is shown. All the trials for which: the fit had a \(\chi^2/ndf < 2\), the efficiency was larger than \(0.4\) 
times the one of the central value, the significance was larger than \(0.5\) times the one of the central value and
the variation of the efficiency was larger than 0.5\% were taken into account. 
The systematic uncertainty was assigned considering the rms and the deviation from unity of these distributions. 
The values are reported in Table~\ref{tab:DsEffSyst010_ml} and Table~\ref{tab:DsEffSyst3050_ml} for central and semi-central events respectively.
\begin{figure}[!h]
  \begin{center}
    \includegraphics[width=0.65\textwidth]{figures/Dsubs/ml/eff_syst_010/CutVarSyst_Ds_010_lowpt_ptbin0}
    \includegraphics[width=0.65\textwidth]{figures/Dsubs/ml/eff_syst_010/CutVarSyst_Ds_010_lowpt_ptbin1}
    \includegraphics[width=0.65\textwidth]{figures/Dsubs/ml/eff_syst_010/CutVarSyst_Ds_010_lowpt_ptbin2}
    \includegraphics[width=0.65\textwidth]{figures/Dsubs/ml/eff_syst_010/CutVarSyst_Ds_010_lowpt_ptbin3}
    \caption{Variations of raw yield, prompt and feed-down \Dsubs{} efficiency, significance, signal-to-background ratio, 
            and corrected yield obtained with the cut-variation study for the \Dsubs{} mesons in the 0--10\% centrality class
            for the 2 to 6 \(\GeV{}/c\) \pt{} region.}
    \label{fig:DsCutSyst010_ml_1}
  \end{center}
\end{figure}
\begin{figure}[!h]
  \begin{center}
    \includegraphics[width=0.65\textwidth]{figures/Dsubs/ml/eff_syst_010/CutVarSyst_Ds_010_lowpt_ptbin4}
    \includegraphics[width=0.65\textwidth]{figures/Dsubs/ml/eff_syst_010/CutVarSyst_Ds_010_highpt_ptbin0}
    \includegraphics[width=0.65\textwidth]{figures/Dsubs/ml/eff_syst_010/CutVarSyst_Ds_010_highpt_ptbin1}
    \includegraphics[width=0.65\textwidth]{figures/Dsubs/ml/eff_syst_010/CutVarSyst_Ds_010_highpt_ptbin2}
    \caption{Variations of raw yield, prompt and feed-down \Dsubs{} efficiency, significance, signal-to-background ratio, 
            and corrected yield obtained with the cut-variation study for the \Dsubs{} mesons in the 0--10\% centrality class
            for the 6 to 24 \(\GeV{}/c\) \pt{} region.}
    \label{fig:DsCutSyst010_ml_2}
  \end{center}
\end{figure}
\begin{figure}[!h]
  \begin{center}
    \includegraphics[width=0.65\textwidth]{figures/Dsubs/ml/eff_syst_010/CutVarSyst_Ds_010_highpt_ptbin3}
    \includegraphics[width=0.65\textwidth]{figures/Dsubs/ml/eff_syst_010/CutVarSyst_Ds_010_highpt_ptbin4}
    \caption{Variations of raw yield, prompt and feed-down \Dsubs{} efficiency, significance, signal-to-background ratio, 
            and corrected yield obtained with the cut-variation study for the \Dsubs{} mesons in the 0--10\% centrality class
            for the 24 to 50 \(\GeV{}/c\) \pt{} region.}
    \label{fig:DsCutSyst010_ml_3}
  \end{center}
\end{figure}

\begin{figure}[!h]
  \begin{center}
    \includegraphics[width=0.65\textwidth]{figures/Dsubs/ml/eff_syst_3050/CutVarSyst_Ds_3050_lowpt_ptbin0}
    \includegraphics[width=0.65\textwidth]{figures/Dsubs/ml/eff_syst_3050/CutVarSyst_Ds_3050_lowpt_ptbin1}
    \includegraphics[width=0.65\textwidth]{figures/Dsubs/ml/eff_syst_3050/CutVarSyst_Ds_3050_lowpt_ptbin2}
    \includegraphics[width=0.65\textwidth]{figures/Dsubs/ml/eff_syst_3050/CutVarSyst_Ds_3050_lowpt_ptbin3}
    \caption{Variations of raw yield, prompt and feed-down \Dsubs{} efficiency, significance, signal-to-background ratio, 
            and corrected yield obtained with the cut-variation study for the \Dsubs{} mesons in the 30--50\% centrality class
            for the 2 to 8 \(\GeV{}/c\) \pt{} region.}
    \label{fig:DsCutSyst3050_ml_1}
  \end{center}
\end{figure}
\begin{figure}[!h]
  \begin{center}
    \includegraphics[width=0.65\textwidth]{figures/Dsubs/ml/eff_syst_3050/CutVarSyst_Ds_3050_hight_ptbin0}
    \includegraphics[width=0.65\textwidth]{figures/Dsubs/ml/eff_syst_3050/CutVarSyst_Ds_3050_hight_ptbin1}
    \includegraphics[width=0.65\textwidth]{figures/Dsubs/ml/eff_syst_3050/CutVarSyst_Ds_3050_hight_ptbin2}
    \includegraphics[width=0.65\textwidth]{figures/Dsubs/ml/eff_syst_3050/CutVarSyst_Ds_3050_hight_ptbin3}
    \caption{Variations of raw yield, prompt and feed-down \Dsubs{} efficiency, significance, signal-to-background ratio, 
            and corrected yield obtained with the cut-variation study for the \Dsubs{} mesons in the 30--50\% centrality class
            for the 8 to 36 \(\GeV{}/c\) \pt{} region.}
    \label{fig:DsCutSyst3050_ml_2}
  \end{center}
\end{figure}

\begin{table}[htbp]
  \begin{center}
   \begin{tabular}{|c|c|}
    \hline
    \textbf{$\mathbf{\pt{}}$ (GeV/\textit{c})} & \textbf{Efficiency syst. unc.} \\
    \hline
    2-3 & 8\%\\
    \hline
    3-4 & 8\%\\
    \hline
    4-5 & 6\%\\
    \hline
    5-6 & 6\%\\
    \hline
    6-8 & 4\%\\
    \hline
    8-12 & 4\%\\
    \hline
    12-16 & 4\%\\
    \hline
    16-24 & 4\%\\
    \hline
    24-36 & 4\%\\
    \hline
    36-50 & 4\%\\ 
    \hline
   \end{tabular}
  \end{center}
  \caption{Systematic uncertainty estimated with the threshold-variation study for the \Dsubs{} mesons in the 0--10\% centrality class.}
  \label{tab:DsEffSyst010_ml}
\end{table}
\begin{table}[htbp]
  \begin{center}
   \begin{tabular}{|c|c|}
    \hline
    \textbf{$\mathbf{\pt{}}$ (GeV/\textit{c})} & \textbf{Efficiency syst. unc.} \\
    \hline
    2-3 & 6\%\\
    \hline
    3-4 & 6\%\\
    \hline
    4-6 & 6\%\\
    \hline
    6-8 & 4\%\\
    \hline
    8-12 & 4\%\\
    \hline
    12-16 & 4\%\\
    \hline
    16-24 & 4\%\\
    \hline
    24-36 & 4\%\\
    \hline
   \end{tabular}
  \end{center}
  \caption{Systematic uncertainty estimated with the threshold-variation study for the \Dsubs{} mesons in the 30--50\% centrality class.}
  \label{tab:DsEffSyst3050_ml}
\end{table}

 \clearpage

\subsubsection{Generated \pt{} shape}
The procedure for the estimation of the systematic uncertainty due to the generated \pt{} shape follow what is performed 
in the standard analysis, which is described in Section~\ref{sec:gen_pt_syst}. \\
The variation of the efficiency of prompt \Dsubs{} mesons is reported in Figure~\ref{fig:genpt_010} and Figure~\ref{fig:genpt_3050}, for the 0-10\% 
and 30-50\% centrality classes respectively.
The assigned systematic are reported in Table~\ref{tab:DsGenPtSyst010_ml} and Table~\ref{tab:DsGenPtSyst3050_ml} for the two centrality classes.
\begin{figure}[htbp]
  \begin{center}
   \includegraphics[width=0.8\textwidth]{figures/Dsubs/ml/SystPtWeights_010_ML_wrtTAMU_LHC19c3a}
  \caption{Prompt \Dsubs{} efficiency (left) with different \pt{} weights and its variation (right) in the 0--10\% centrality
           class.}
  \label{fig:genpt_010}
  \end{center}
\end{figure}
\begin{figure}[htbp]
  \begin{center}
   \includegraphics[width=0.8\textwidth]{figures/Dsubs/ml/SystPtWeights_3050_ML_wrtTAMU_LHC19c3b}
  \caption{Prompt \Dsubs{} efficiency (left) with different \pt{} weights and its variation (right) in the 30--50\% centrality
           class.}
  \label{fig:genpt_3050}
  \end{center}
\end{figure}
\begin{table}[htbp]
  \begin{center}
   \begin{tabular}{|c|c|}
    \hline
    \textbf{$\mathbf{\pt{}}$ (GeV/\textit{c})} & \textbf{Efficiency syst. unc.} \\
    \hline
    2-3 & 5\%\\
    \hline
    3-5 & 3\%\\
    \hline
    5-12 & 2\%\\
    \hline
    12-50 & 0\%\\
    \hline
   \end{tabular}
  \end{center}
  \caption{Systematic uncertainty due to the generated \pt{} shape for the \Dsubs{} mesons in the 0--10\% centrality class.}
  \label{tab:DsGenPtSyst010_ml}
\end{table}
\begin{table}[htbp]
\begin{center}
  \begin{tabular}{|c|c|}
  \hline
  \textbf{$\mathbf{\pt{}}$ (GeV/\textit{c})} & \textbf{Efficiency syst. unc.} \\
  \hline
  2-3 & 5\%\\
  \hline
  3-6 & 3\%\\
  \hline
  6-12 & 1\%\\
  \hline
  12-36 & 0\%\\
  \hline
  \end{tabular}
\end{center}
\caption{Systematic uncertainty due to the generated \pt{} shape for the \Dsubs{} mesons in the 30--50\% centrality class.}
\label{tab:DsGenPtSyst3050_ml}
\end{table}

\subsection{Results}
The \(\RAA{}\) obtained is reported in Figure~\ref{fig:raa_ml} for the two centrality classes considered and compared to the results of the 
standard analysis. The correction for the imperfect description of the TPC dead zones is included in both the measurements. 
\begin{figure}[htbp]
  \begin{center}
   \includegraphics[width=0.48\textwidth]{figures/Dsubs/ml/Comp_Raa_010}
   \includegraphics[width=0.48\textwidth]{figures/Dsubs/ml/Comp_Raa_3050} 
  \caption{\pt{}-differential \(\RAA{}\) of prompt \Dsubs{} in the 0-10\% (left panel) and 30-50\% (right panel) centrality classes 
           measured with the BDT approach (red) compared to the measurement obtained with the linear selections (blue).}
  \label{fig:raa_ml}
  \end{center}
\end{figure}

\clearpage

\subsection{Checks on topological and PID variables}
Given the complex nature of the selection applied by the BDT model it is interesting to investigate their effect. To this end the distributions 
of the variables used in the training before, with the pre-selections applied, and after the BDT selections employed for the analysis are considered. \\
In Figure~\ref{fig:topo_1D_3050_pt3_4} and Figure~\ref{fig:topo_2D_3050_pt3_4} the distributions of topological and kinematic variables of prompt \Dsubs{} candidate, from the \texttt{LHC19d4a} production, are reported fom the 30--50\% centrality class 
in the \pt{} range between 3 and 4 \(\GeV{}/c\) (the amount of data in \(2 < \pt{} < 3~\GeV{}/c\) is too small to have clear information). \\
\begin{figure}[htbp]
  \begin{center}
   \includegraphics[width=\textwidth]{figures/Dsubs/ml/LHC19d4b_TopolVars_pT3-4_1D}
  \caption{Distributions of \pt{}, kinematic and topological variables before (blue) and after the BDT selection (red) for prompt \Dsubs{} candidates in \(3 < \pt{} < 4~\GeV{}/c\) and 30--50\% centrality class.}
  \label{fig:topo_1D_3050_pt3_4}
  \end{center}
\end{figure}
\begin{figure}[htbp]
  \begin{center}
   \includegraphics[width=\textwidth]{figures/Dsubs/ml/LHC19d4b_TopolVars_pT3-4_2D}
  \caption{Distributions of \pt{}, kinematic and topological variables versus \pt{} (top) and BDT (bottom) output before (blue) and after the BDT selection (red) for prompt \Dsubs{} candidates in \(3 < \pt{} < 4~\GeV{}/c\) and 30--50\% centrality class.}
  \label{fig:topo_2D_3050_pt3_4}
  \end{center}
\end{figure}
Differently to the linear selections, reported in Table~\ref{tab:dsubs2} for the 30-50\% centrality class, it looks like that the BDT applies smoother selections on the candidate variables. The \pt{} distribution is almost not affected by the selection. 
Looking at the 2-D plots versus \pt{} no clear transverse momentum dependence of the selections is present. However, it must be considered that the selections of the BDT could not be well 
represented in these low-dimensional plots. \\
In Figure~\ref{fig:PID_1D_3050_pt3_4},~\ref{fig:PID_2D_3050_pt3_4_TPCTOF} and~\ref{fig:PID_2D_3050_pt3_4} and the distributions of PID variables of prompt \Dsubs{} candidate are reported fom the 30--50\% centrality class 
in the \pt{} range between 3 and 4 \(\GeV{}/c\). \\
\begin{figure}[htbp]
  \begin{center}
   \includegraphics[width=0.85\textwidth]{figures/Dsubs/ml/LHC19d4b_PIDVars_pT3_4_TPC}
   \includegraphics[width=0.85\textwidth]{figures/Dsubs/ml/LHC19d4b_PIDVars_pT3_4_TOF}
   \includegraphics[width=0.85\textwidth]{figures/Dsubs/ml/LHC19d4b_PIDVars_pT3_4_Comb}
  \caption{\(n\sigma\) distributions for TPC (top), TOF (middle) and the combination used in the training (bottom) before (blue) and after the BDT selection (red) for prompt \Dsubs{} candidates in \(3 < \pt{} < 4~\GeV{}/c\) and 30--50\% centrality class.}
  \label{fig:PID_1D_3050_pt3_4}
  \end{center}
\end{figure}
\begin{figure}[htbp]
  \begin{center}
   \includegraphics[width=\textwidth]{figures/Dsubs/ml/LHC19d4b_PIDVars_pT3_4_TPCTOF_2D}
  \caption{Scatter plots of TPC and TOF \(n\sigma\) distributions for the different daughters and particle hypothesis before (blue) and after the BDT selection (red) for prompt \Dsubs{} candidates in \(3 < \pt{} < 4~\GeV{}/c\) and 30--50\% centrality class.}
  \label{fig:PID_2D_3050_pt3_4_TPCTOF}
  \end{center}
\end{figure}
\begin{figure}[htbp]
  \begin{center}
   \includegraphics[width=0.85\textwidth]{figures/Dsubs/ml/LHC19d4b_PIDVars_pT3_4_TPC_2D}
   \includegraphics[width=0.85\textwidth]{figures/Dsubs/ml/LHC19d4b_PIDVars_pT3_4_TOF_2D}
  \caption{Scatter plots of \(n\sigma\) distributions of the first and third \Dsubs{} daughter for TPC (top) and TOF (bottom) before (blue) and after the BDT selection (red) for prompt \Dsubs{} candidates in \(3 < \pt{} < 4~\GeV{}/c\) and 30--50\% centrality class.}
  \label{fig:PID_2D_3050_pt3_4}
  \end{center}
\end{figure}
Looking at the PID distributions and in particular to the compatibility with the kaon hypothesis for the middle daughter, that is always a kaon, it is possible to notice that the 
conservative PID imposed in the pre-selections is almost preserved by the BDT selections. 